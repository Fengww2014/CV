\resheading{项目经历}
  \begin{itemize}[leftmargin=*]
\item 
 \ressubsingleline{基于P4语言的网络转发协议开发}{可编程协议,P4,SDN}{2019.04 --至今}
{\small
	\begin{itemize}
		\item 开发环境: P4网络编程语言,SDN,mininet
		\item 项目属性:网络协议,SDN,自定义网络协议的开发
		\item 项目简介: 着手关注软件定义网络,学习P4语言,并开发新的集支持智能转发,缓存,计算等为一体的网络协议和模型,撰写一篇会议论文:\\
		\href{https://rec.ustc.edu.cn/share/338739d0-1595-11ea-8267-e347179c193a}{\textit{Implementing ICN over P4 in HTTP Scenario}}(2019 IEEE International Conference on Hot Information-Centric Networking (HotICN))\\
	\end{itemize}
}    
\item 
 \ressubsingleline{基于NDN的移动自组织网络研究}{网络协议,ICN,智能路由}{2018.04 -- 2019.04}
{\small
   	\begin{itemize}
   		\item 项目属性:无线网络领域,无线自组织网络
   		\item 项目简介: 就NDN网络搭建无线自组织网络,提出路由算法和多转发策略,并就NDN的网内缓存特点提出基于NDN的任务执行方案,撰写相关专利一篇和两篇会议文章:\\
   		1. 计算机网络中基于边缘计算的任务执行方法,申请号:201910899720.9 \\
   		2. \href{https://ieeexplore.ieee.org/document/8605969} {\textit{MANET for Disaster Relief based on NDN}}(2018 IEEE International Conference on Hot Information-Centric Networking (HotICN))\\
   		3. \href{https://dl.acm.org/citation.cfm?doid=3341568.3342110}{\textit{Scheduling of Distributed Collaborative Tasks on NDN based MANET}(ACM SIGCOMM 2019 Workshop on Mobile Air­Ground Edge Computing, Systems, Networks, and Applications)}
   	\end{itemize}
}
\item
  \href{https://upload-images.jianshu.io/upload_images/15436989-3949bacbca4e82e4.JPG?imageMogr2/auto-orient/strip%7CimageView2/2/w/1240}{\textbf{multi-modal flexibility of image - guided needle puncture robot system, municipal project}}
  \ressubsingleline{} {嵌入式,机械,电气,图像引导}{2016.09 -- 2017.05(本科)}
 
  {\small
  \begin{itemize}
    \item 项目属性:医疗电气领域,图像引导机器人,大学生创新创业北京市级项目
    \item 项目简介: 大学生创新创业项目,通过设计该机电系统,通过影像和压力等反馈,远程控制穿刺动作。获得北京市合格证书,并参加校内创新展览,撰写一篇国内期刊文章:\\
    \href{https://kns.cnki.net/KCMS/detail/detail.aspx?dbcode=CJFQ&dbname=CJFDLAST2018&filename=TXRG201704001&v=MDk3NTJUM3FUcldNMUZyQ1VSTE9lWitScUZ5bmdXcnZLTVRYWmFiRzRIOWJNcTQ5RlpZUjhlWDFMdXhZUzdEaDE=}{\textit{非对准情况下静脉穿刺针力测量[J].透析与人工器官,2017,28(04):1-5.}} 
  \end{itemize}
  }
\end{itemize}